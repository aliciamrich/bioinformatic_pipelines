% load packages
\usepackage{geometry}
\usepackage{xcolor}
\usepackage{eso-pic}
\usepackage{fancyhdr}
\usepackage{sectsty}
\usepackage{fontspec}
\usepackage{titlesec}

%% Set page size with a wider right margin
\geometry{
  letterpaper,
  left=10mm,
  right=60mm,
  top=20mm,
  bottom=20mm
}

\usepackage{graphicx}
\graphicspath{%
  {_extensions/nrennie/PrettyPDF/}%
  {../_extensions/nrennie/PrettyPDF/}%
  {../../_extensions/nrennie/PrettyPDF/}%
  {../../../_extensions/nrennie/PrettyPDF/}%
  {.}%
}

% expose geometry margins as usable lengths
\makeatletter
\newlength{\RightMargin}\setlength{\RightMargin}{\Gm@rmargin}
\newlength{\TopMargin}  \setlength{\TopMargin}{\Gm@tmargin}
\makeatother

% --- right margin bar + logo positioned from real margins ---
\usepackage{eso-pic,xcolor,graphicx}
\definecolor{light}{HTML}{E6E6FA}
\definecolor{highlight}{HTML}{800080}
\definecolor{dark}{HTML}{330033}

% --- tweakables ---
\newlength{\BarWidth}       \setlength{\BarWidth}{3cm}      % <-- bar thickness you want
\newlength{\LogoWidth}      \setlength{\LogoWidth}{1.8cm}   % <-- logo width
\newlength{\LogoPadTop}     \setlength{\LogoPadTop}{8mm}    % <-- gap below top margin

% computed positions
\newlength{\BarX}   \setlength{\BarX}{\dimexpr\paperwidth-\BarWidth\relax}
\newlength{\LogoX}  \setlength{\LogoX}{\dimexpr\paperwidth-\BarWidth/2-\LogoWidth/2\relax}
\newlength{\LogoY}  \setlength{\LogoY}{\dimexpr\paperheight-\TopMargin-\LogoPadTop\relax}

% --- right bar + logo ---
\AddToShipoutPictureBG{%
  % bar: fixed width (not the whole margin)
  \AtPageLowerLeft{%
    \put(\LenToUnit{\BarX},0){%
      \color{light}\rule{\LenToUnit{\BarWidth}}{\LenToUnit\paperheight}%
    }%
  }%
  % logo: centered in the bar, near top. No color applied to keep original look.
  \AtPageLowerLeft{%
    \put(\LenToUnit{\LogoX},\LenToUnit{\LogoY}){%
    \includegraphics[width=\LogoWidth]{logo.png}%
    }%
  }%
}


% footer aligned with text block (not in the margin bar)
\usepackage{fancyhdr}
\fancypagestyle{mystyle}{
  \fancyhf{}
  \renewcommand\headrulewidth{0pt}
  \fancyfoot[R]{\thepage}
  % no offset
}
\pagestyle{mystyle}

% ensure header/footer width equals the text width
\makeatletter
\setlength{\headwidth}{\textwidth}
\makeatother

\setlength{\footskip}{20pt}

\usepackage{etoolbox}
\usepackage{ragged2e} % for robust ragged-right
\AtBeginEnvironment{figure}{\RaggedRight}

\let\OldFigure\figure
\let\EndOldFigure\endfigure
\renewenvironment{figure}{\OldFigure\RaggedRight}{\EndOldFigure}

% overhang by an explicit factor (e.g., 1.20\textwidth)
\newcommand{\overhangimage}[2][1.20\textwidth]{%
  \noindent\makebox[\textwidth][l]{\includegraphics[width=#1]{#2}}%
}

% overhang exactly into the right margin (fill text width + margin)
\newcommand{\overhangtorightmargin}[1]{%
  \noindent\makebox[\textwidth][l]{%
    \includegraphics[width=\dimexpr\textwidth+\RightMargin\relax]{#1}%
  }%
}


%% style the chapter/section fonts
\chapterfont{\color{dark}\fontsize{20}{16.8}\selectfont}
\sectionfont{\color{dark}\fontsize{20}{16.8}\selectfont}
\subsectionfont{\color{dark}\fontsize{14}{16.8}\selectfont}
\titleformat{\subsection}
  {\sffamily\Large\bfseries}{\thesection}{1em}{}[{\titlerule[0.8pt]}]

% left align title
\makeatletter
\renewcommand{\maketitle}{\bgroup\setlength{\parindent}{0pt}
\begin{flushleft}
  {\sffamily\huge\textbf{\MakeUppercase{\@title}}} \vspace{0.3cm} \newline
  {\Large {\@subtitle}} \newline
  \@author
\end{flushleft}\egroup
}
\makeatother

% --- robust path detection for bundled Ubuntu fonts ---
\newcommand\UBUNTUPATH{}%

\IfFileExists{_extensions/nrennie/PrettyPDF/Ubuntu/Ubuntu-Regular.ttf}{%
  \renewcommand\UBUNTUPATH{_extensions/nrennie/PrettyPDF/Ubuntu/}%
}{%
  \IfFileExists{../_extensions/nrennie/PrettyPDF/Ubuntu/Ubuntu-Regular.ttf}{%
    \renewcommand\UBUNTUPATH{../_extensions/nrennie/PrettyPDF/Ubuntu/}%
  }{%
    \IfFileExists{../../_extensions/nrennie/PrettyPDF/Ubuntu/Ubuntu-Regular.ttf}{%
      \renewcommand\UBUNTUPATH{../../_extensions/nrennie/PrettyPDF/Ubuntu/}%
    }{%
      \IfFileExists{../../../_extensions/nrennie/PrettyPDF/Ubuntu/Ubuntu-Regular.ttf}{%
        \renewcommand\UBUNTUPATH{../../../_extensions/nrennie/PrettyPDF/Ubuntu/}%
      }{%
        \renewcommand\UBUNTUPATH{}% fallback: use system-installed fonts
      }%
    }%
  }%
}

% if bundled fonts are found, use Path=..., else use system fonts
\IfFileExists{\UBUNTUPATH Ubuntu-Regular.ttf}{%
  \setsansfont{Ubuntu}[
    Path=\UBUNTUPATH,
    Scale=0.9,
    Extension=.ttf,
    UprightFont=*-Regular,
    BoldFont=*-Bold,
    ItalicFont=*-Italic,
  ]
  \setmainfont{Ubuntu}[
    Path=\UBUNTUPATH,
    Scale=0.9,
    Extension=.ttf,
    UprightFont=*-Regular,
    BoldFont=*-Bold,
    ItalicFont=*-Italic,
  ]
}{%
  \setsansfont{Ubuntu}[Scale=0.9]
  \setmainfont{Ubuntu}[Scale=0.9]
}

% gt_packages
\usepackage{longtable,booktabs,array,multirow,wrapfig,float}
\usepackage{pdflscape,threeparttable,threeparttablex,makecell}
\usepackage{colortbl}
\usepackage{caption}
\usepackage{pdfpages}

